The Python 3 implementation of this simulation is available in a GitHub repository\footnote{Avalable at : \url{https://github.com/Teusner/Tether\_Modeling}}. This code is based on Numpy~\cite{numpy}, Matplotlib~\cite{matplotlib} and Scipy~\cite{scipy} packages. The goal of this simulator is to study the viability of such a system, in particular to validate the performance of the tether with this behavioral model.

So we will create a class \textit{TetherElement} which will represent a node. It will have to contain its mass, volume and distance information from its neighbors, but also its position, velocity and acceleration, as well as a pointer to each of its two neighbors. Finally it is necessary to have each coefficient $K_p$, $K_d$ and $K_i$ to compute $\overrightarrow{F_{t, p}}$ and $\overrightarrow{F_{t, n}}$.

This will allow us later on to implement a \textit{Tether} class to be able to simulate a tether. This object must have a length, a number of elements and a list containing the different \textit{TetherElement} that compose it. It must also have the mass and the volume of each node, but also the length of each link between nodes in order to correctly instantiate each \textit{TetherElement}.

A diagram of these two classes is visible on the \textsc{Figure}~\ref{fig:uml}. It respects the \textsc{UML} format and allows to see the different class variables and methods associated to each class.

\begin{figure}[!htb]
    \centering
    \resizebox{0.50\textwidth}{!}{
        \begin{tikzpicture}
            \begin{class}[text width=6cm]{Tether}{0,0}
                \attribute{+ element\_mass : double}
                \attribute{+ element\_volume : double}
                \attribute{+ element\_length : double}
                \attribute{+ position\_first : numpy.ndarray}
                \attribute{+ position\_last : numpy.ndarray}
                \attribute{+ elements : list of \textit{TetherElement}}
            \end{class}
        
            \begin{class}[text width=6cm]{TetherElement}{8.5,0}
                \attribute{+ mass : double}
                \attribute{+ volume : double}
                \attribute{+ length : double}
                \attribute{+ position : numpy.ndarray}
                \attribute{+ velocity : numpy.ndarray}
                \attribute{+ acceleration : numpy.ndarray}
                \attribute{+ previous : TetherElement}
                \attribute{+ next : TetherElement}
                \attribute{+ K\_p : double}TetherElement
                \attribute{+ K\_d : double}
                \attribute{+ K\_i : double}
                \operation{+ F\_p(self) : numpy.ndarray}
                \operation{+ F\_b(self) : numpy.ndarray}
                \operation{+ F\_f(self) : numpy.ndarray}
                \operation{+ Ft\_prev(self) : numpy.ndarray}
                \operation{+ Ft\_next(self) : numpy.ndarray}
            \end{class}
        
            \aggregation{Tether}{}{~~~n}{TetherElement}
        \end{tikzpicture}
    }
    \caption{UML diagram of the \textit{Tether} and \textit{TetherElement} classes}
    \label{fig:uml}
\end{figure}

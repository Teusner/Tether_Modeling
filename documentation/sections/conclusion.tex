In conclusion, the tether modeling as presented in this paper seems to give correct results. The idea of reasoning by finite differences allows us to simulate the tether as a succession of tether elements linked by variable-length links. The problem induced by this method is to provide a model to describe the force of a node on its neighbors to reach the target length. The proposed behavioral model provides good results here, and it leads to a simulator with a physical meaning.

Besides, there are still some issues that have not been addressed. First of all the tether was not tested in an environment where the extremities were subjected to movement. Second, the Tether is simulated with a fixed length, which is not necessarily the case during a submarine mission. Indeed it is common to have to unroll and rewind the Tether during the mission to prevent it from becoming tangled. Then, no force simulates the stiffness of the Tether, but it is not infinitely flexible. A solution to the two previous problems seems to be presented in~\cite{ganoni_unreal}. Finally, there is no transmissible torque across the tether, which can typically be induced by a tether twist.
Simulation in the field of robotics is a powerful tool. Indeed, it allows to easily and quickly test the robot in different conditions and to have a reproducibility of the results. Then it let us be able to create situations that would be difficult to find in reality, in order to make sure of the robot's behavior. It should be noticed that the simulation of robots does not replace tests in real conditions, but it remains practical during the development phase. However, the simulation of robots in the maritime environment is a field that still has shortcomings, especially when we want to simulate the tethers of submarine robots. Indeed, it is not easy to find an analytical way to simulate a tether that does not require large resources, especially when simulation environments become complex. This is why we will try to suggest a finite difference simulation of the tether by proposing a behavioral model of force between each tether element. The results seem satisfactory and with a correct initialization of the position of the tether elements, the behavior of the tether seems quite right. However, the tether stiffness phenomena are not taken into account in this modeling, the tether is of fixed length, and both ends have been fixed so the system has not been tested dynamically.